\documentclass[11pt,a4paper]{article}

\usepackage[utf8]{inputenc}
\usepackage[T1]{fontenc}
\usepackage{amsmath,amssymb,amsthm}
\usepackage{hyperref}
\usepackage{url}
\usepackage{graphicx}
\usepackage{xcolor}
\usepackage{tcolorbox}
\usepackage{geometry}
\geometry{margin=1in}

% Non-claims box style
\newtcolorbox{nonclaims}{
  colback=red!5!white,
  colframe=red!50!black,
  title={\textbf{Non-Claims}},
  fonttitle=\bfseries
}

% Theorem environments
\theoremstyle{definition}
\newtheorem{definition}{Definition}[section]
\theoremstyle{plain}
\newtheorem{theorem}{Theorem}[section]
\newtheorem{lemma}[theorem]{Lemma}

\title{A Debt Ledger Model for Tracking Proof Obligations\\in the Prime Counting Function}
\author{RH Debt Ledger Project}
\date{\today}

\begin{document}

\maketitle

\begin{abstract}
We present a \emph{debt ledger} model for organizing and verifying proof obligations
related to bounds on the prime counting function $\pi(x)$. Rather than claiming
any new mathematical result, we describe a verification infrastructure that
(1)~makes explicit which claims have been mechanically verified,
(2)~isolates unpaid obligations equivalent in difficulty to the Riemann Hypothesis,
and (3)~provides reproducible artifact chains for all paid rungs.
The tail-bound obligation remains unpaid. This paper makes no claim regarding RH.
\end{abstract}

\tableofcontents
\newpage

\section{Introduction}
\label{sec:introduction}

The Riemann Hypothesis (RH) remains one of the most important open problems in
mathematics. Any approach to RH must carefully distinguish between what has been
established and what remains conjectural. This paper describes a software
infrastructure---a \emph{debt ledger}---designed to make such distinctions explicit
and mechanically verifiable.

\begin{nonclaims}
This paper:
\begin{itemize}
  \item Is \textbf{not} a proof of RH.
  \item Is \textbf{not} a new bound on $\pi(x)$ or $\psi(x)$.
  \item Is \textbf{not} verification of RH beyond known computational checks.
  \item Does \textbf{not} provide ``evidence suggesting RH is true.''
  \item Makes \textbf{no forward projection} about future progress.
\end{itemize}
The tail-bound obligation (Section~\ref{sec:limits}) remains unpaid and is
equivalent in difficulty to RH itself.
\end{nonclaims}

\subsection{Motivation}

When working on difficult mathematical problems, it is easy to lose track of
which steps are rigorously established and which contain gaps. The debt ledger
model addresses this by:
\begin{enumerate}
  \item Recording each proof step as a ``rung'' with explicit contracts.
  \item Distinguishing between \emph{paid} obligations (mechanically verified)
        and \emph{unpaid} obligations (equivalent to the core conjecture).
  \item Providing deterministic verification scripts that anyone can run.
\end{enumerate}

\subsection{Contributions}

We contribute:
\begin{enumerate}
  \item A deterministic verification ladder (R4--R22) with reproducible releases.
  \item Explicit isolation of the tail-bound obligation as the unpaid debt.
  \item Cryptographic receipt chains (SHA-256, GPG, post-quantum hybrid signatures).
  \item Fully reproducible release artifacts built from git tags.
\end{enumerate}

See \texttt{RUNG\_INDEX.md} for the complete rung map and \texttt{STATUS.md} for
the current ledger state.

\section{Problem Statement}
\label{sec:problem}

\subsection{The Prime Counting Function}

Let $\pi(x)$ denote the number of primes not exceeding $x$. The Prime Number
Theorem establishes that
\[
  \pi(x) \sim \frac{x}{\log x} \quad \text{as } x \to \infty.
\]
A more refined approximation is provided by the logarithmic integral:
\[
  \operatorname{li}(x) = \int_0^x \frac{dt}{\log t}.
\]

\subsection{Connection to the Riemann Hypothesis}

The Riemann zeta function $\zeta(s)$ is defined for $\Re(s) > 1$ by
\[
  \zeta(s) = \sum_{n=1}^{\infty} \frac{1}{n^s}
\]
and admits analytic continuation to $\mathbb{C} \setminus \{1\}$. The Riemann
Hypothesis asserts that all non-trivial zeros of $\zeta(s)$ have real part
$\frac{1}{2}$.

RH is equivalent to the bound
\[
  \psi(x) = x + O(\sqrt{x} \log^2 x),
\]
where $\psi(x) = \sum_{p^k \le x} \log p$ is the Chebyshev function.

\subsection{The Verification Challenge}

Any claimed bound on $\pi(x) - \operatorname{li}(x)$ must either:
\begin{enumerate}
  \item Assume RH (in which case the assumption must be stated explicitly), or
  \item Prove the required zero-free region (equivalent to proving RH for
        the purpose of deriving the bound).
\end{enumerate}

The debt ledger model makes this choice explicit by recording which claims
are paid (unconditional) and which carry unpaid debt (conditional on RH or
equivalent).

\section{The Debt Ledger Model}
\label{sec:ledger}

\subsection{Core Concepts}

The debt ledger model organizes proof obligations into three categories:

\begin{definition}[Paid Obligation]
An obligation is \emph{paid} if it can be mechanically verified from artifacts
in the repository without assuming any unproven conjecture.
\end{definition}

\begin{definition}[Unpaid Obligation]
An obligation is \emph{unpaid} if its satisfaction requires proving a statement
equivalent in difficulty to RH (or another open problem).
\end{definition}

\begin{definition}[Rung]
A \emph{rung} is a discrete step in the verification ladder, consisting of:
\begin{itemize}
  \item A contract specifying what the rung establishes and what it does NOT claim.
  \item Artifacts providing evidence (equations, digests, signatures).
  \item A verifier script that checks the rung's integrity.
  \item A release tag for reproducibility.
\end{itemize}
\end{definition}

\subsection{The NA0 Debt Framing}

The key insight is that any bound of the form
\[
  |\pi(x) - \operatorname{li}(x)| < f(x)
\]
for all $x > x_0$ requires controlling the contribution of zeta zeros.
The ``tail'' contribution from zeros with large imaginary part cannot be bounded
without RH or an equivalent statement.

We call this the \textbf{NA0 debt}: the obligation to bound the tail contribution
from infinitely many zeros. This debt remains unpaid.

See \texttt{proof\_artifacts/R7\_BOUND\_STATEMENT/R7\_BOUND\_STATEMENT.md} for
the explicit statement of the bound and its dependencies.

\subsection{Artifact Chain}

Each rung produces artifacts that feed into subsequent rungs:
\begin{enumerate}
  \item R4--R6: Equation inventory, error bound components, parameter choices.
  \item R7: Explicit bound statement citing R4--R6.
  \item R8: Regeneration of equation inventory for comparison.
  \item R9: Input manifest with frozen digests.
  \item R10: Assembly receipt (root hash over all inputs).
  \item R11--R18: Signature layers (GPG, post-quantum hybrid).
  \item R20: Deterministic rebuild of assembly root.
  \item R21: Deterministic rebuild of release zip.
  \item R22: Rung index and status ledger.
\end{enumerate}

See \texttt{RUNG\_INDEX.md} for the complete map.

\section{The Verification Ladder}
\label{sec:ladder}

The verification ladder consists of rungs R4--R22. Each rung has a contract,
artifacts, and a verifier script. We summarize the key rungs below.

\subsection{Foundation Rungs (R4--R6)}

\textbf{R4: Transfer Repro.} Records the equation inventory for partial summation
transfer. Does not claim proof correctness.
See \texttt{proof\_artifacts/R4\_TRANSFER\_REPRO/07\_EQUATION\_INVENTORY.md}.

\textbf{R5: Error Bound Source.} Provides a menu of error bound components.
Does not claim the bounds are valid or optimal.
See \texttt{proof\_artifacts/R5\_ERROR\_BOUND\_SOURCE/R5\_MENU.md}.

\textbf{R6: Instantiation.} Records specific parameter choices.
Does not claim optimality.
See \texttt{proof\_artifacts/R6\_INSTANTIATION/R6\_INSTANTIATION\_RECORD.md}.

\subsection{Bound Statement (R7)}

\textbf{R7: Bound Statement.} Provides an explicit statement of the bound,
citing R4--R6 for its components. Does NOT claim RH.
See \texttt{proof\_artifacts/R7\_BOUND\_STATEMENT/R7\_BOUND\_STATEMENT.md}.

The bound statement explicitly records which terms are paid (verified)
and which carry unpaid debt (the tail contribution).

\subsection{Integrity Rungs (R8--R10)}

\textbf{R8: Comparison Run.} Regenerates the equation inventory and compares
to the frozen version. Detects any drift.
See \texttt{proof\_artifacts/R8\_COMPARISON\_RUN/R8\_COMPARISON\_RECEIPT.md}.

\textbf{R9: No-Surprise Assembly.} Freezes the input manifest with SHA-256
digests. Ensures no hidden inputs.
See \texttt{proof\_artifacts/R9\_NO\_SURPRISE\_ASSEMBLY/R9\_INPUT\_MANIFEST.txt}.

\textbf{R10: Assembly Receipt.} Computes a canonical root hash over all
proof artifacts. Forms the signing payload.
See \texttt{proof\_artifacts/R10\_ASSEMBLY\_RECEIPT/R10\_RECEIPT.json}.

\subsection{Signature Rungs (R11, R14--R18)}

\textbf{R11: GPG Signature.} Optional detached GPG signature over R10 assembly root.
See \texttt{proof\_artifacts/R11\_SIGNATURE/sigs/dave.asc}.

\textbf{R14: Key Capture.} Records GPG fingerprints for attribution.
See \texttt{proof\_artifacts/R14\_KEY\_CAPTURE/captures/dave.fingerprint.txt}.

\textbf{R15--R17: Post-Quantum Signatures.} Provides hybrid PQ signatures
(P-384 + ML-DSA-65) for quantum-resistant attestation.
See \texttt{proof\_artifacts/R15\_PQ\_SIGNATURE/sigs\_pq/dave.p384\_mldsa65.sig}.

\subsection{Reproducibility Rungs (R20--R22)}

\textbf{R20: Assembly Root Rebuild.} Deterministically rebuilds R10 from the
verify surface. Proves the root hash is reproducible.
See \texttt{proof\_artifacts/R20\_REBUILD\_ASSEMBLY\_ROOT/R20\_CONTRACT.md}.

\textbf{R21: Release Zip Rebuild.} Deterministically rebuilds the release zip
from a git tag. Ensures byte-identical releases.
See \texttt{proof\_artifacts/R21\_RELEASE\_ZIP\_REBUILD/R21\_CONTRACT.md}.

\textbf{R22: Rung Index.} Provides the canonical map of all rungs and
current ledger status.
See \texttt{RUNG\_INDEX.md} and \texttt{STATUS.md}.

\section{Results So Far}
\label{sec:results}

We summarize what the verification ladder has established.

\subsection{Paid Obligations}

The following obligations have been paid (mechanically verified):

\begin{enumerate}
  \item \textbf{Classical signature (GPG):} Dave's detached signature over the
        R10 assembly root.
        See \texttt{proof\_artifacts/R11\_SIGNATURE/sigs/dave.asc}.

  \item \textbf{Post-quantum hybrid signature:} P-384 + ML-DSA-65 signature
        over the assembly root.
        See \texttt{proof\_artifacts/R15\_PQ\_SIGNATURE/sigs\_pq/}.

  \item \textbf{Assembly root rebuild:} Deterministic regeneration of R10
        from the verify surface.
        See \texttt{VERIFY\_R20\_REBUILD\_ASSEMBLY\_ROOT.sh}.

  \item \textbf{Release reproducibility:} Deterministic zip rebuild from git tag.
        See \texttt{scripts/rebuild\_release\_zip.sh}.

  \item \textbf{Key fingerprint capture:} GPG fingerprint recorded for attribution.
        See \texttt{proof\_artifacts/R14\_KEY\_CAPTURE/captures/}.

  \item \textbf{PQ tooling receipt:} Backend and version captured for PQ signatures.
        See \texttt{proof\_artifacts/R16\_PQ\_TOOLING/}.
\end{enumerate}

\subsection{Verification Coverage}

The main verifier runs 23 steps:
\begin{itemize}
  \item Steps 1--11: Repository integrity (git status, required files, redaction,
        exhibits, contribution ledger).
  \item Steps 12--23: Rung verifiers (R4--R20).
\end{itemize}

To run verification:
\begin{verbatim}
VR_STRICT=1 ./VERIFY.sh
\end{verbatim}

Expected output:
\begin{verbatim}
=== VERIFICATION PASSED ===
\end{verbatim}

\subsection{What This Means}

The verification ladder establishes that:
\begin{enumerate}
  \item The equation inventory, error bounds, and parameter choices are
        frozen and reproducible.
  \item The assembly root hash is deterministic.
  \item Signed attestations are cryptographically bound to the artifacts.
  \item Release zips can be rebuilt identically from git tags.
\end{enumerate}

This provides \emph{infrastructure} for tracking proof obligations, but does
NOT prove any mathematical claim about $\pi(x)$ or RH.

\section{Limits and Remaining Obligations}
\label{sec:limits}

This section explicitly states what remains unpaid and why.

\subsection{The Tail Bound Obligation}

The central unpaid obligation is the \textbf{tail bound}: controlling the
contribution of zeta zeros $\rho = \beta + i\gamma$ with $|\gamma| > T$
for any fixed $T$.

Any bound of the form
\[
  |\psi(x) - x| < C \sqrt{x} \log^2 x
\]
requires summing over \emph{all} zeros. The sum converges, but its size depends
on the location of zeros. Without knowing that all zeros lie on the critical
line, the tail cannot be bounded effectively.

\textbf{Status:} UNPAID. Equivalent in difficulty to RH.

See \texttt{proof\_artifacts/R7\_BOUND\_STATEMENT/R7\_BOUND\_STATEMENT.md}
for the explicit statement of where this debt enters.

\subsection{Remaining Obligations}

\begin{center}
\begin{tabular}{|l|l|l|}
\hline
\textbf{Item} & \textbf{Status} & \textbf{Notes} \\
\hline
Tail bound proof & UNPAID & The dragon is boxed, not slain \\
RH claim & NOT MADE & This repo makes no claim regarding RH \\
Proof reproduction & NOT DONE & No proofs are reproduced \\
External audit & NOT DONE & Third-party review pending \\
\hline
\end{tabular}
\end{center}

\subsection{Why This Matters}

The debt ledger model makes explicit a common source of errors in mathematical
claims: conflating ``we have organized the pieces'' with ``we have proven the
result.'' By explicitly recording unpaid obligations, we prevent scope creep
and premature claims.

The infrastructure is valuable even with unpaid debt because it:
\begin{enumerate}
  \item Provides a clear target for future work.
  \item Ensures that any future claim of progress can be verified against
        the frozen artifacts.
  \item Maintains an audit trail for reproducibility.
\end{enumerate}

\subsection{What Would Pay the Debt}

The tail bound obligation would be paid by either:
\begin{enumerate}
  \item A proof of RH (establishing all zeros have $\beta = \frac{1}{2}$).
  \item An unconditional zero-density estimate sufficient for the required bound.
  \item A fundamentally different approach that avoids the explicit sum over zeros.
\end{enumerate}

None of these is claimed or implied by this work.

\section{Reproducibility}
\label{sec:reproducibility}

A key design goal is that all artifacts can be independently reproduced.

\subsection{Deterministic Builds}

Release zips are built deterministically:
\begin{itemize}
  \item Fixed file timestamps (2024-01-01 00:00:00 UTC).
  \item Stable file ordering (LC\_ALL=C sort).
  \item No extended attributes (stripped on macOS).
  \item No zip timestamps (zip -X flag).
\end{itemize}

To rebuild a release:
\begin{verbatim}
./scripts/rebuild_release_zip.sh <tag> /tmp/rebuilt.zip
\end{verbatim}

The resulting SHA-256 should match the published release.

\subsection{Assembly Root Rebuild}

The R10 assembly root can be regenerated from the verify surface:
\begin{verbatim}
./scripts/rebuild_r10_assembly_root.sh
\end{verbatim}

The output should match \texttt{proof\_artifacts/R10\_ASSEMBLY\_RECEIPT/R10\_RECEIPT.json}.

See \texttt{VERIFY\_R20\_REBUILD\_ASSEMBLY\_ROOT.sh} for the automated check.

\subsection{Verification Commands}

\textbf{Baseline verification (23 steps):}
\begin{verbatim}
VR_STRICT=1 ./VERIFY.sh
\end{verbatim}

\textbf{Full GPG verification (requires pubkey import):}
\begin{verbatim}
gpg --import <pubkey>
gpg --verify proof_artifacts/R11_SIGNATURE/sigs/dave.asc \
    proof_artifacts/R11_SIGNATURE/R10_ASSEMBLY_ROOT.txt
\end{verbatim}

\textbf{Full PQ verification (requires oqsprovider):}
\begin{verbatim}
OPENSSL_MODULES=/path/to/ossl-modules \
  ./scripts/verify_pq_sig.sh p384_mldsa65 \
  proof_artifacts/R15_PQ_SIGNATURE/sigs_pq/dave.p384_mldsa65.sig \
  proof_artifacts/R11_SIGNATURE/R10_ASSEMBLY_ROOT.txt \
  proof_artifacts/R15_PQ_SIGNATURE/pubkeys/dave.p384_mldsa65.pub
\end{verbatim}

\subsection{Checkpoint Releases}

Each rung has an associated git tag and GitHub release:
\begin{itemize}
  \item Tag format: \texttt{r<N>-<name>}
  \item Asset: \texttt{r<N>-<name>-verify-surface.zip}
  \item Contains: Verifier scripts, proof artifacts, documentation.
\end{itemize}

See \texttt{RUNG\_INDEX.md} for the complete list of tags.

\section{Conclusion}
\label{sec:conclusion}

We have presented a debt ledger model for organizing proof obligations related
to bounds on the prime counting function. The key contributions are:

\begin{enumerate}
  \item A deterministic verification ladder (R4--R22) with explicit contracts
        for each rung.
  \item Clear separation between paid obligations (mechanically verified) and
        unpaid obligations (equivalent to RH).
  \item Reproducible release artifacts with cryptographic attestations.
  \item An explicit statement of remaining debt: the tail bound is unpaid.
\end{enumerate}

\subsection{What This Work Is}

\begin{itemize}
  \item A framework for tracking proof obligations.
  \item A demonstration of reproducible mathematical artifact pipelines.
  \item An honest accounting of what has and has not been established.
\end{itemize}

\subsection{What This Work Is NOT}

\begin{itemize}
  \item A proof of RH.
  \item A claim that RH is true or false.
  \item Evidence suggesting RH is more or less likely.
  \item A substitute for peer review of mathematical content.
\end{itemize}

\subsection{Future Work}

The debt ledger infrastructure supports future work by:
\begin{enumerate}
  \item Providing a clear target (pay the tail bound obligation).
  \item Ensuring any claimed progress can be verified against frozen artifacts.
  \item Maintaining an audit trail for reproducibility.
\end{enumerate}

The dragon is boxed, not slain. The box is well-documented.


\bibliographystyle{plain}
\bibliography{bib}

\end{document}
