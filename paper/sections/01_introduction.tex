\section{Introduction}
\label{sec:introduction}

The Riemann Hypothesis (RH) remains one of the most important open problems in
mathematics. Any approach to RH must carefully distinguish between what has been
established and what remains conjectural. This paper describes a software
infrastructure---a \emph{debt ledger}---designed to make such distinctions explicit
and mechanically verifiable.

\begin{nonclaims}
This paper:
\begin{itemize}
  \item Is \textbf{not} a proof of RH.
  \item Is \textbf{not} a new bound on $\pi(x)$ or $\psi(x)$.
  \item Is \textbf{not} verification of RH beyond known computational checks.
  \item Does \textbf{not} provide ``evidence suggesting RH is true.''
  \item Makes \textbf{no forward projection} about future progress.
\end{itemize}
The tail-bound obligation (Section~\ref{sec:limits}) remains unpaid and is
equivalent in difficulty to RH itself.
\end{nonclaims}

\subsection{Motivation}

When working on difficult mathematical problems, it is easy to lose track of
which steps are rigorously established and which contain gaps. The debt ledger
model addresses this by:
\begin{enumerate}
  \item Recording each proof step as a ``rung'' with explicit contracts.
  \item Distinguishing between \emph{paid} obligations (mechanically verified)
        and \emph{unpaid} obligations (equivalent to the core conjecture).
  \item Providing deterministic verification scripts that anyone can run.
\end{enumerate}

\subsection{Contributions}

We contribute:
\begin{enumerate}
  \item A deterministic verification ladder (R4--R22) with reproducible releases.
  \item Explicit isolation of the tail-bound obligation as the unpaid debt.
  \item Cryptographic receipt chains (SHA-256, GPG, post-quantum hybrid signatures).
  \item Fully reproducible release artifacts built from git tags.
\end{enumerate}

See \texttt{RUNG\_INDEX.md} for the complete rung map and \texttt{STATUS.md} for
the current ledger state.
