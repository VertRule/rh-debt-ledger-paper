\section{Problem Statement}
\label{sec:problem}

\subsection{The Prime Counting Function}

Let $\pi(x)$ denote the number of primes not exceeding $x$. The Prime Number
Theorem establishes that
\[
  \pi(x) \sim \frac{x}{\log x} \quad \text{as } x \to \infty.
\]
A more refined approximation is provided by the logarithmic integral:
\[
  \operatorname{li}(x) = \int_0^x \frac{dt}{\log t}.
\]

\subsection{Connection to the Riemann Hypothesis}

The Riemann zeta function $\zeta(s)$ is defined for $\Re(s) > 1$ by
\[
  \zeta(s) = \sum_{n=1}^{\infty} \frac{1}{n^s}
\]
and admits analytic continuation to $\mathbb{C} \setminus \{1\}$. The Riemann
Hypothesis asserts that all non-trivial zeros of $\zeta(s)$ have real part
$\frac{1}{2}$.

RH is equivalent to the bound
\[
  \psi(x) = x + O(\sqrt{x} \log^2 x),
\]
where $\psi(x) = \sum_{p^k \le x} \log p$ is the Chebyshev function.

\subsection{The Verification Challenge}

Any claimed bound on $\pi(x) - \operatorname{li}(x)$ must either:
\begin{enumerate}
  \item Assume RH (in which case the assumption must be stated explicitly), or
  \item Prove the required zero-free region (equivalent to proving RH for
        the purpose of deriving the bound).
\end{enumerate}

The debt ledger model makes this choice explicit by recording which claims
are paid (unconditional) and which carry unpaid debt (conditional on RH or
equivalent).
