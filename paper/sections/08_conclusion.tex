\section{Conclusion}
\label{sec:conclusion}

We have presented a debt ledger model for organizing proof obligations related
to bounds on the prime counting function. The key contributions are:

\begin{enumerate}
  \item A deterministic verification ladder (R4--R22) with explicit contracts
        for each rung.
  \item Clear separation between paid obligations (mechanically verified) and
        unpaid obligations (equivalent to RH).
  \item Reproducible release artifacts with cryptographic attestations.
  \item An explicit statement of remaining debt: the tail bound is unpaid.
\end{enumerate}

\subsection{What This Work Is}

\begin{itemize}
  \item A framework for tracking proof obligations.
  \item A demonstration of reproducible mathematical artifact pipelines.
  \item An honest accounting of what has and has not been established.
\end{itemize}

\subsection{What This Work Is NOT}

\begin{itemize}
  \item A proof of RH.
  \item A claim that RH is true or false.
  \item Evidence suggesting RH is more or less likely.
  \item A substitute for peer review of mathematical content.
\end{itemize}

\subsection{Future Work}

The debt ledger infrastructure supports future work by:
\begin{enumerate}
  \item Providing a clear target (pay the tail bound obligation).
  \item Ensuring any claimed progress can be verified against frozen artifacts.
  \item Maintaining an audit trail for reproducibility.
\end{enumerate}

The dragon is boxed, not slain. The box is well-documented.
